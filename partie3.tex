\section{Parser spécifique}

Un des points crucial de \emph{Kevoree-c} est la possibilité de reconstruire un état mémoire depuis un fichier représentant un modèle. Cette phase de dé-serialisation repose sur une partie critique du programme. En effet la version actuelle utilisée pour les expériences, en plus de ne fonctionner qu'avec une version fixe de \emph{KMF}, a été réalisée dans des contraintes de temps forte et ne reconnait qu'une partie minimale du méta-modèle. Le code produit ne suivant pas le patron de conception et les raisons énoncées le rendent difficilement maintenable.

Pour palier à ces défauts et mieux comprendre les mécanismes internes de \emph{Kevoree-c} j'ai donc commencé à étudier les possibilités de ré-écriture du dé-serializer.

Il devait répondre aux contraintes suivantes:

\begin{itemize}
\item avoir une faible empreinte mémoire RAM,
\item avoir le moins de traitement ad-hoc à la structure de données,
\item être facile à relire et modifier
\end{itemize}

Mon premier réflexe a été de vouloir utiliser un générateur de parser, de cette manière seule la grammaire aurait été à écrire, le parser en \emph{C} aurait été généré par l'outil.

\subsection{AntlrV3}
AntlrV4 est le générateur de parser utilisé pendant le module de compilation d'ESIR2, dans sa version 4 il ne permet pas/plus de générer du \emph{C} c'est pourquoi j'ai testé AntlrV3.

Pour vite avoir une idée du coût mémoire d'Antlr j'ai utilisé une simple grammaire reconnaissant le format JSON\footnote{https://github.com/antlr/grammars-v4/blob/master/json/JSON.g4} en partant d'un exemple de projet \emph{C} pour AntlrV3\footnote{https://github.com/antlr/examples-v3/tree/master/C}.

Malheureusement même avec une grammaire reconnaissant uniquement un fichier \emph{JSON} et n'effectuant aucune action le coût mémoire est trop important. Sans compter le poids du parser généré l'ensemble des fichiers à inclure pour le faire fonctionner pèse déjà plus de $400Kb$.

\subsection{Flex et Bison}
Quand on parle de générateur de parser on pense très souvent à \emph{Flex}\cite{flex}, pour l'analyse lexicale et à \emph{Bison}\cite{bison} pour l'analyse syntaxique

Ensemble ils peuvent produire du code \emph{C} de taille cette fois bien plus raisonnable. En suivant la même démarche que pour AntlrV3 j'ai d'abord créé un parser pour uniquement reconnaître la syntaxe JSON.

\subsubsection{Portage et entrées/sorties}

Le parser lit son fichier d'entrée à l'aide des méthodes \emph{fopen} et \emph{fclose} manipulant des types \emph{C} : \emph{FILE}. Le problème étant que \emph{Contiki} n'utilise pas ce standard d'entrées/sorties mais des interfaces spécialisées pour son système de fichiers.

\emph{Flex} et \emph{Bison} ne proposent aucune option pour changer la façon dont le parser va lire le fichier d'entrée, de plus le code généré est tellement optimisé que le modifier directement est très difficile et long.

\subsubsection{Portage et tableaux}

Réutiliser des parties du code généré c'est également avéré être une tâche impossible.

% limites du runtime C

\subsection{Solution sur mesure}

% problèmes de mémoires

\section{Compilation}

\subsection{L'intérêt du générateur du code}

\subsection{Une structure de compilateur}

\cite{freeman2004head} \cite{Appel2003MCI599718}

\subsection{Générer une structure de données}

Les choix techniques

Diagramme UML simplifié, en annexe peut être le complet de Kevoree, risque d'être dur à passer en format A4

\subsection{Générer les outils de manipulation}

serialization, dé-serialization

\subsection{Limites et avenir}

Confiance dans le compilateur, les quelques cas ad-hoc




