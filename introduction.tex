\section*{Introduction} % Pas de numérotation
\phantomsection
\addcontentsline{toc}{section}{Introduction}

\subsection{L'IRISA}

L'\emph{IRISA}\footnote{https://www.irisa.fr/}, Institut de recherche en informatique et systèmes aléatoires, est un laboratoire de recherche créée en 1975. Formé de 41 équipes réparties sur 4 pôles en Bretagne il se focalise sur «la recherche en recherche en informatique, automatique, traitement du signal et des images».

\subsection{L'équipe \diver}

L'équipe \diver\footnote{http://diverse.irisa.fr/}, anciennement \emph{Triskell}, est dirigée par Benoit \textsc{Baudry}. Elle est composée de près de 40 personnes, dont 7 permanentes, et se focalise sur la diversité dans le génie logiciel. Cette ligne directrice se traduit par des travaux de recherche dans les langages, la variabilité, l'adaptation et la diversification des logiciels.

\subsection{Le stage}
Ce stage est réalisé en vue de la validation de ma deuxième année à l'ESIR. À la frontière entre le monde des systèmes embarqués et celui du développement dirigé par les modèles l'objectif principal du stage est développer des outils plus génériques que ceux déjà réalisés dans le cadre de thèses réalisées au sein de l'équipe \diver.

Le choix d'un centre de recherche est motivé par une future année en double diplôme ESIR3 / Master Recherche en Informatique ainsi qu'une éventuelle poursuite en thèse.