\section{L'Internet des Objets}

L'Internet des objets, abrégé en IoT, ou Cyber Physical Systems est un réseau non pas composé d'ordinateurs ou de serveurs mais de nombreux composants bien plus petits par leurs tailles et limités par leurs caractéristiques.

Ce nouveau type de réseau apporte de nouvelles contraintes pour le développement de logiciel en particulier.

Ces contraintes vont de la limitation de la puissance de calcul, du stockage limité, de la durée de vie de la batterie et jusqu'à la complexité de gestion d'un réseau distribué.

\section{L'environnement Contiki}

Les limites physiques des nœuds composant les réseaux étudiés font qu'un système d'exploitation spécifique doit être utilisé en lieu et place du classique GNU/Linux.

%photo de la board

Les nœuds principalement utilisés pour les expériences sont des M3, globalement les nœuds sont sur une architecture ARM, le processeur est cadencé à 72MHz, possède 64kB de RAM et 16MB de ROM.

Ces nœuds sont également équipés de différents capteurs tels que:
\begin{itemize}
\item un capteur de luminosité,
\item un thermomètre,
\item un baromètre,
\item un accéléromètre/magnétomètre,
\item un gyromètre
\end{itemize}

\subsection{Principales différences}

De par 
limites du système, le débug, le runtime C, le système de fichiers


\section{Testbed iot-lab}


intéret du testbed

limites du testbed

\section{model@runtime, Kevoree et KMF}

\subsection{model@runtime}

Le besoin de reconfigurer des réseaux.

Chargement de binaire, ELF loader

résumé du papier

\subsection{KMF}

inspiré d'EMF mais pour leurs besoins

\subsection{Kevoree}

les différentes versions

surtout faire la transition sur le besoin de produire une implémentation depuis un modèle de Kevoree écrit en KMF