%\setcounter{section}{0}
\section{Premier développement sur Contiki}
Pour mieux appréhender la variété de l'environnement et des outils la première tâche qui m'a été confiée n'était pas au cœur de ses systèmes. J'ai tout d'abord lu l'article \citep{acostapadilla} pour me familiariser avec cet environnement. 

Les nœuds \emph{m3} étant uniquement capable d'exécuter du code déjà compilé il a fallu que je mette en place un environnement dit de \emph{Cross-compilation}, ou compilation croisée, qui consiste à produire un code binaire exécutable par une machine ayant une architecture différente de celle servant au développement.

Pour développer un nouveau module pour \emph{Contiki} il suffit d'en récupérer les sources\footnote{https://github.com/iot-lab/contiki}, de se placer dans le répertoire contenant les sources d'un module et de le compiler en précisant l'architecture cible, iotlab-m3.

\lstset{language=bash, captionpos=b, caption=Résumé des étapes de compilation}
\begin{lstlisting}[frame=single]
git clone https://github.com/iot-lab/contiki
cd contiki/example/hello-world
make TARGET=iotlab-m3
\end{lstlisting}

Cette compilation produira un firmware contenant \emph{Contiki}, le module courant ainsi que tout autre module marqué en dépendance. Ce firmware pourra directement être flashé sur un nœud \emph{m3}.

\section{Compression des modèles}

Pour pouvoir s'adapter et selon le principe du \emph{model@runtime} chaque nœud d'un réseau IoT doit comparer le nouveau modèle du réseau avec le courant et en déduire s'il doit se modifier. Il est donc nécessaire que le nouveau modèle soit échangé entre tous les nœuds. Dans un soucis de temps de développement et puisque cela n'impacte pas l'idée de \emph{model@runtime} l'implémentation de \emph{kevoree-c} ne compresse pas les modèles avant de les transmettre aux pairs voisins.

\subsection{État de l'art}

Puisque aucun algorithme de compression de données n'est fourni avec \emph{Contiki} j'ai d'abord commencé par en chercher implémentés en \emph{C}, recommendés pour les systèmes embarqués et disponibles en sources libres dans des licences permettant leurs réutilisations.

L'objectif étant d'avoir le meilleur taux de compression tout en restant le moins demandeur de ressources matérielles, en particulier en mémoire RAM.

Les implémentations retenues et étudiées sont \emph{Huffman}, \emph{fastlz}\footnote{http://fastlz.org/} et \emph{lzf} \footnote{http://oldhome.schmorp.de/marc/liblzf.html}.

Les résultats ont été obtenus en utilisant l'outil \emph{Massif}\footnote{http://valgrind.org/docs/manual/ms-manual.html} de la suite d'outils Valgrind. \emph{Massif} permet de mesurer l'utilisation de différentes zones mémoire lors de l'exécution d'un programme. Par défaut seul l'utilisation du tas est mesuré mais la pile peut également l'être après configuration. Les fichiers de données produits sont binaires mais peuvent être visualisés à l'aide de l'outil \emph{msprint} fourni avec \emph{Massif}.

Les mesures étant faites sur un ordinateur classique d'architecture 64bit et non sur l'environnement de production la compilation a été réalisée en 32bit pour réduire la taille des adresses de pointeur manipulées et ainsi avoir des résultats plus fiables.



% Peut être mettre l'image en annexe
\begin{figure}[ht!]
\centering
\includegraphics[scale=0.7]{images/msprint.png}
\caption{Exemple d'extrait de sortie de Massif, ici l'utilisation maximum de RAM est au snapshot 45 et est d'environ 400KB}
\end{figure}


% Compilation des résultats en bar plot

\subsection{Tentatives de portage}

trop de limite, 

\subsection{Table d'association}

Re blabla sur les limites du systèms

les différentes approches envisagées 

le tableau comparatif des implém existantes
Juste un graphe des ratios et des utilisations de ressources
En annexe le détail des résultats mesurés.


La méthode retenue,

le package contiki crée


\section{Parser spécifique}

\section{Compilation}

Le gros du stage pour le moment

\subsection{L'intérêt de générer du code}

\subsection{Un vrai compilateur et pas juste une transcription de modèle}