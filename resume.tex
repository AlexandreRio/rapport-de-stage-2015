%\section*{Résumé}
%\phantomsection
\pagestyle{empty}
\addcontentsline{toc}{section}{Résumé}

\vspace*{\fill}
\begin{center}

Le \emph{model@runtime} est un paradigme visant à permettre l'adaptation et la modification de réseaux de machines au travers de modèles. Si cela fonctionne déjà sur des réseaux de serveurs des thèses s'efforcent de démontrer que le cas est également possible pour l'Internet des Objets, où les machines sont bien plus limitées en terme de puissance de calcul et d'espace mémoire.

Ce stage s'inscrit dans une de ces thèses et proposent une implémentation d'un compilateur lisant un méta-modèle décrivant un programme implémentant le paradigme du \emph{model@runtime} et produisant un programme équivalent pour fonctionner sur un réseau de capteurs.
\end{center}
\vspace*{\fill}